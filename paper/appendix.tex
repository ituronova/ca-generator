%%%%%%%%%%%%%%%%%%%%%%%%%%%%%%%%%%%%%%%%%%%%%%%%%%%%%%%%%%%%%%%%%%%%%%%%%%%%%%%
\section{Correctness of Conditional Partial Derivatives}
%%%%%%%%%%%%%%%%%%%%%%%%%%%%%%%%%%%%%%%%%%%%%%%%%%%%%%%%%%%%%%%%%%%%%%%%%%%%%%%

%==============================================================================
\subsection{Validity of Memories and Visibility of Counters}
%==============================================================================

% We are using the parametric definition below with conditional derivatives.

We need the following additional notions in order to reason about correctness of
the construction of CAs from regexes via conditional partial derivatives.
%
Let $R$ be a normalized regex. 
%
A counter $X$ \emph{is visible in} $R$, denoted
$X\in\VISIBLE{R}$, if either $R=YZ$ and $X=Y$, or else if
$X$ does not occur in $Y$ and $X$ is visible in $Z$, i.e.,
$X\in\VISIBLE{Z}\setminus\COUNTERS{Y}$. In other words,
%
\[
\VISIBLE{R}
=
\left\{
\begin{array}{ll}
  \emptyset, &\textrm{if $R=\emp$;}\\
  \{S\{\ell,k\}\}\cup(\VISIBLE{Z}\setminus\COUNTERS{S}), &\textrm{else if $R=S\{\ell,k\}Z$;}\\
  \VISIBLE{Z}\setminus\COUNTERS{Y}, & \textrm{otherwise, where $R=YZ$.}
\end{array}
\right.
\]
Let
$\HIDDEN{R}\eqdef\COUNTERS{R}\setminus\VISIBLE{R}$.  A counter memory
$\mem$ is \emph{valid for} $R$ if $\mem(X)=0$ for all
$X\in\HIDDEN{R}$, otherwise $\mem$ is \emph{invalid} for $R$.
Intuitively, every hidden counter in $R$ must have the initial value
0 in any valid memory.  We only consider counter memories $\mem$ that
are valid for $R$ in the context of $\LANG{\mem}{R}$.
%\lh{I was lost before Margus gave me some intuition motivating the Scp. Maybe some of it could go before or after the formal def of Scp. 
%My understanding now is that Scp is trying to capture the set of counters that can be non-zero in memories at this point, memories that are produced by starting from the original regex with the all-zero memory. Scp is only (over?)approximating this set of potentially non-zero counters. For understanding the definition, it is also needed to understand that in $R = YZ$, $Y$ can be an inner loop within the outer loop $Z$, and the $YZ$ could have been obtained by unfolding $Z$ once when computing a derivative, so both counters $Y$ and $Z$ can be non-zero. The definition was not making sense to me before realizing this.}
%\mv{but I think this explanation is too early here, there is the chicken-egg problem}

\begin{ex}
  Let $X=\texttt{a\{3\}}$. Then $X$ is visible in $X\cdot X*$ but hidden
  in $X{*}\cdot X$. A counter memory $\mem$ such that $\mem(X)=2$ is valid for $X\cdot X*$
  but invalid for $X{*}\cdot X$.
\end{ex}

\begin{ex}
Let $X$ be the regex \texttt{(a(bc)\{7\}d)\{8\}}. Then $X$ and $Y =
\texttt{(bc)\{7\}}$ are both counters in $X$ but only $X$ is visible
in $X$.  Now consider the regex $Y\texttt{d}X$, or more precisely
$Y\cdot (\texttt{d}\cdot X)$ to emphasize the normalized form.  In
this case $\VISIBLE{Y\texttt{d}X}=\{X,Y\}$.  The visible counters of
$\texttt{c}Y\texttt{d}X$ are $\{X,Y\}$.  In fact, as shown below,
$\spair{\IncrOp[X]}{Y\texttt{d}X}$ is the (conditional)
$\texttt{a}$-derivative of $X$ and
$\spair{\IncrOp[Y]}{\texttt{c}Y\texttt{d}X}$ is the
$\texttt{b}$-derivative of $Y\texttt{d}X$.
These are in fact the only possible regexes that arise here through derivation starting with $X$.
The \texttt{c}-derivative of $\texttt{c}Y\texttt{d}X$ is
$\spair{\blambda x.x}{Y\texttt{d}X}$, and the \texttt{d}-derivative of
$Y\texttt{d}X$ is $\spair{\ExitOp[Y]}{X}$. All other derivatives are empty.
\end{ex}

We use the following lemma in the correctness theorem of partial derivatives.
If $E$ is a set of counters, then $\RESET{E}$ resets the values of all counters
in $E$ to 0. Observe that $\RESET{\{c\}}$ is in general different from
$\ExitOp[c]$ because $\ExitOp[c](\mem)=\bot$ when $\mem\not\models\CanExit{c}$.

In the proof of the lemma (and also multiple further proofs), we will need the
notion of the \emph{size} of a regex $R$, denoted by $\SIZEOF{R}$, which
corresponds to the number of nodes in the abstract syntax tree of $R$
(apart from the case when $R$ is $\eps$) and which we define as follows:

\[
\begin{array}{l}
\SIZEOF{\emp}=0\quad \SIZEOF{\pred{\alpha}}=1 \quad
\SIZEOF{R_1\cdot R_2} = \SIZEOF{R_1}+\SIZEOF{R_2}+1\quad \\
    \SIZEOF{R_1 | R_2} = \SIZEOF{R_1}+\SIZEOF{R_2}+1\quad
    \SIZEOF{R\{n,m\}}=\SIZEOF{R}+1\quad
    \SIZEOF{R{*}}=\SIZEOF{R}+1\quad
\end{array}
\]

\begin{lemma}
\label{lemma:RESET}
  If $X$ and $Y$ are normalized regexes and $\mem$ is valid for $XY$ then
  \em
\[
\LANG{\mem}{XY} = \LANG{\mem}{X}\cdot\LANG{\RESET{\VISIBLE{X}}(\mem)}{Y}.
\]
\end{lemma}
\begin{proof}
  By induction over the pair $(\SIZEOF{X},n)$ where $n=\MAX{c}-\mem(c)$ if $X$
  starts with a counter $c$ or else $n=0$. Let $(m',n')<(m,n)$ iff either $m' < m$
  or else $m'=m$ and $n'<n$.
 
  \paragraph{Base case $X=\emp$}
  Then $\RESET{\VISIBLE{\eps}}(\mem)=\mem$
  and $\LANG{\mem}{\emp}=\{\eps\}$.
  
  \paragraph{Induction case $X=S\{\ell,k\}Z$}
  Let $c = S\{\ell,k\}$, $\mem_0=\ExitOp[c](\mem)$ and $\mem_1=\IncrOp[c](\mem)$.

    \begin{itemize}

    \item Case $\mem(c)=\MAX{c}$.
        Then $\mem_1=\bot$ and $\mem_0$ is
  valid for $ZY$ because $\mem$ is valid for $XY$ and $\mem_0(c)=0$.
  (Observe that if $d\in\HIDDEN{ZY}$
  then either $\mem_0(d)=0$ if $d=c$ or else $\mem(d)=0$ because then
  $d\in\HIDDEN{XY}$.)
  It follows that
  \begin{eqnarray*}
    \LANG{\mem}{XY} &\stackrel{(\mem_1=\bot)}{=}& \LANG{\mem_0}{ZY} \\
    &\stackrel{\textrm{(IH)}}{=}&
    \LANG{\mem_0}{Z}\cdot\LANG{\RESET{\VISIBLE{Z}}(\mem_0)}{Y} \\
    &\stackrel{(\mem_1=\bot)}{=}&
    \LANG{\mem}{X}\cdot\LANG{\RESET{\VISIBLE{Z}}(\ExitOp[c](\mem))}{Y} \\
    &=&
    \LANG{\mem}{X}\cdot\LANG{\RESET{\VISIBLE{X}}(\mem)}{Y} 
  \end{eqnarray*}
  where the last equality holds because $\VISIBLE{X}=\{c\}\cup(\VISIBLE{Z}\setminus\COUNTERS{S})$ and
  $\mem(d)=0$ for all $d\in\COUNTERS{S}$ because $\mem$ is valid for $X$.
  Hence $\RESET{\VISIBLE{Z}}(\ExitOp[c](\mem)) = \RESET{\VISIBLE{X}}(\mem)$.
  
    \item Case $\mem(c)<\MIN{c}$:
  Then $\mem_0=\bot$.
%  Also $S$ is not nullable because $X$ is normalized and $\ell > 0$.
  Here
  $\mem_1$ is valid for $XY$ because $\mem$ is valid and $c\in\VISIBLE{XY}$.
  Also, the (IH) applies because $k-\mem_1(c) < k-\mem(c)$.
  \begin{eqnarray*}
    \LANG{\mem}{XY} &\stackrel{(\mem_0=\bot)}{=}&
    \LANG{\mem}{S}\cdot\LANG{\mem_1}{XY}\\
    &\stackrel{\textrm{(IH)}}{=}&
    \LANG{\mem}{S}\cdot\LANG{\mem_1}{X}\cdot\LANG{\RESET{\VISIBLE{X}}(\mem_1)}{Y} \\
    &\stackrel{(c\in\VISIBLE{X})}{=}&
    \LANG{\mem}{S}\cdot\LANG{\mem_1}{X}\cdot\LANG{\RESET{\VISIBLE{X}}(\mem)}{Y} \\
    &\stackrel{(\mem_0=\bot)}{=}&
    \LANG{\mem}{X}\cdot\LANG{\RESET{\VISIBLE{X}}(\mem)}{Y}
  \end{eqnarray*}

\item Case $\MIN{c} \leq \mem(c) < \MAX{c}$.
  Then $\mem_0\neq\bot$ and $\mem_1\neq\bot$.
  This case is a combination of the above two cases where the IH is applied twice
  under similar conditions.
  \begin{eqnarray*}
    \LANG{\mem}{XY} &=&
    \LANG{\mem}{S}\cdot\LANG{\mem_1}{XY}\cup\LANG{\mem_0}{ZY}\\
    &\stackrel{\textrm{($2\times$IH)}}{=}&
    \LANG{\mem}{S}\cdot\LANG{\mem_1}{X}\cdot\LANG{\RESET{\VISIBLE{X}}(\mem_1)}{Y}\cup
    \LANG{\mem_0}{Z}\cdot\LANG{\RESET{\VISIBLE{Z}}(\mem_0)}{Y} \\
     &=&
    \LANG{\mem}{S}\cdot\LANG{\mem_1}{X}\cdot\LANG{\RESET{\VISIBLE{X}}(\mem)}{Y}\cup 
    \LANG{\mem_0}{Z}\cdot\LANG{\RESET{\VISIBLE{X}}(\mem)}{Y} \\
    &=&
    \LPAR \LANG{\mem}{S}\cdot\LANG{\mem_1}{X}\cup 
    \LANG{\mem_0}{Z}\RPAR \cdot\LANG{\RESET{\VISIBLE{X}}(\mem)}{Y} \\
    &=&
    \LANG{\mem}{X}\cdot\LANG{\RESET{\VISIBLE{X}}(\mem)}{Y}
  \end{eqnarray*}

  
    \end{itemize}
    
    \paragraph{Induction case $X=\psi Z$}    Trivially $\VISIBLE{X}=\VISIBLE{Z}$.
    \begin{eqnarray*}
      \LANG{\mem}{XY} &=& \den{\psi}\cdot \LANG{\mem}{ZY} \\
      &\stackrel{\textrm{(IH)}}{=}&
      \den{\psi}\cdot \LANG{\mem}{Z}\cdot\LANG{\RESET{\VISIBLE{Z}}(\mem)}{Y} \\
    &=&
    \LANG{\mem}{X}\cdot\LANG{\RESET{\VISIBLE{X}}(\mem)}{Y}
    \end{eqnarray*}

    \paragraph{Induction case $X=(A|B)Z$} Let $X_1=AZ$ and $X_2=BZ$.
    Here $\VISIBLE{X}=\VISIBLE{Z}\setminus\COUNTERS{A|B}$.
    Thus, for all $c\in\COUNTERS{A|B}$, $\mem(c)=0$ because
    $\mem$ is valid for $X$. Therefore,
    if $c\in\VISIBLE{X_i}$ then either $c\in\VISIBLE{X}$ or else
    $c\in\COUNTERS{A|B}$ and $\mem(c)=0$. Hence
    $\RESET{\VISIBLE{X_i}}(\mem) = \RESET{\VISIBLE{X}}(\mem)$.
    \begin{eqnarray*}
      \LANG{\mem}{XY} &=& \LANG{\mem}{X_1Y} \cup  \LANG{\mem}{X_2Y} \\
      &\stackrel{\textrm{($2\times$IH)}}{=}&
      \LANG{\mem}{X_1}\cdot\LANG{\RESET{\VISIBLE{X_1}}(\mem)}{Y}\cup
      \LANG{\mem}{X_2}\cdot\LANG{\RESET{\VISIBLE{X_2}}(\mem)}{Y}\\
      &=&
      \LANG{\mem}{X_1}\cdot\LANG{\RESET{\VISIBLE{X}}(\mem)}{Y}\cup
      \LANG{\mem}{X_2}\cdot\LANG{\RESET{\VISIBLE{X}}(\mem)}{Y}\\
      &=&
      (\LANG{\mem}{X_1}\cup
      \LANG{\mem}{X_2})\cdot\LANG{\RESET{\VISIBLE{X}}(\mem)}{Y}\\
    &=&
    \LANG{\mem}{X}\cdot\LANG{\RESET{\VISIBLE{X}}(\mem)}{Y}
    \end{eqnarray*}

    \paragraph{Induction case $X=S{*}Z$}
    Clearly $\mem$ is valid for $Z$ because it is valid for $X$.
    Here $\VISIBLE{X}=\VISIBLE{Z}\setminus\COUNTERS{S}$.
    So if $c\in\VISIBLE{Z}$ then either $c\in\VISIBLE{X}$ or else 
    $\mem(c)=0$. Thus
    $\RESET{\VISIBLE{Z}}(\mem) = \RESET{\VISIBLE{X}}(\mem)$.
        \begin{eqnarray*}
      \LANG{\mem}{XY} &=& \LANG{\mem}{S}^* \cdot\LANG{\mem}{ZY} \\
      &\stackrel{\textrm{(IH)}}{=}&
      \LANG{\mem}{S}^* \cdot
      \LANG{\mem}{Z}\cdot\LANG{\RESET{\VISIBLE{Z}}(\mem)}{Y}
      \\
      &=&
      \LANG{\mem}{S}^* \cdot
      \LANG{\mem}{Z}\cdot\LANG{\RESET{\VISIBLE{X}}(\mem)}{Y}
      \\
      &=&
    \LANG{\mem}{X}\cdot\LANG{\RESET{\VISIBLE{X}}(\mem)}{Y}
    \end{eqnarray*}
The statement follows by the induction principle.
\end{proof}

%==============================================================================
\subsection{Proof of Theorem~\ref{thm:LANG}}\label{app:thmLANG}
%==============================================================================

First note that $\memInit$ is trivially valid for any regex $R$.
%
Let $S\{0,0\}\eqdef\emp$ and let $m\ominus n \eqdef \max(m-n,0)$.
%
The following property is used below.  

\begin{lemma}
  \label{lemma:COUNTING}
  Let $X=S\{\ell,k\}$ be a normalized counting loop and let $\mem$ be valid for $X$. Then
  $$
   \LANG{\mem}{X} = \bigcup_{i=\ell\ominus \mem(X)}^{k-\mem(X)}\LANG{\memInit}{S}^{(i)}.
  $$
\end{lemma}
\begin{proof}
    By induction over $k-n$ where
   $n=\mem(X)$. Let $\mem_1=\IncrOp[X](\mem)$. Let $L=\LANG{\memInit}{S}$.

  \paragraph{Base case $n=k$}
  Then $\mem_1=\bot$ and so
  $\LANG{\mem}{X}=\LANG{\ExitOp[X](\mem)}{\emp} = \{\eps\}$
  because $\ell \leq k$ and so $\mem\models\CanExit{X}$ and, by definition,
  $L^{(0)}=\{\eps\}$ for any $L$.

  \paragraph{Induction case $n<k$}
   Here $m_1\neq\bot$ and $\LANG{\mem}{S}=\LANG{\memInit}{S} = L$ because $\VISIBLE{X}=\{X\}$.
\begin{eqnarray*}
  \LANG{\mem}{X} &=&  \LANG{\mem}{S}\cdot\LANG{\mem_1}{X}\cup\LANG{\ExitOp[X](\mem)}{\emp} \\
  &=& L\cdot\LANG{\mem_1}{X}\cup\{\eps\mid \ell\leq n\} \\
  &\stackrel{\textrm{(IH)}}{=}&
  L\cdot
  \LPAR
  \bigcup_{i=\ell\ominus (n+1)}^{k-(n+1)}L^{(i)}
  \RPAR
  \cup\{\eps\mid \ell\leq n\}\\
  &=&
  \LPAR
  \bigcup_{i=(\ell\ominus (n+1))+1}^{k-n}L^{(i)}
  \RPAR
  \cup\{\eps\mid \ell\leq n\} \quad = \quad
  \bigcup_{i=\ell\ominus n}^{k-n}L^{(i)}
\end{eqnarray*}
The last two equalities use standard rules of set theory and theory of sequences.
\end{proof}

We can now prove Theorem~\ref{thm:LANG}.

\begin{proof}
  By induction over $\SIZEOF{R}$.
  The base case $R=\emp$ is trivial.
  The main induction case is
  $R=S\{\ell,k\}Z$. Then
  \begin{eqnarray*}
    \LANG{\memInit}{R}&\stackrel{(\textrm{Lemma~\ref{lemma:RESET}})}{=}&
    \LANG{\memInit}{S\{\ell,k\}} \cdot
    \LANG{\memInit}{Z} \\
    &\stackrel{(\textrm{Lemma~\ref{lemma:COUNTING}})}{=}&
    \LPAR \bigcup_{i=\ell}^{k}\LANG{\memInit}{S}^{(i)}\RPAR \cdot
    \LANG{\memInit}{Z}
    \\
    &\stackrel{(2\times \textrm{IH})}{=} &
    \LPAR \bigcup_{i=\ell}^{k}\langof{S}^{(i)}\RPAR \cdot
    \langof{Z} \\
    &=&
    \langof{S\{\ell,k\}}\cdot\langof{Z} \\
    &=&
    \langof{R}
  \end{eqnarray*}
  The remaining cases follow by induction.
\end{proof}

%==============================================================================
\subsection{Proof of Theorem~\ref{theorem:CD}\label{app:thmCD}}
%==============================================================================

\begin{proof}
  By induction over $\SIZEOF{R}$.
 
  \paragraph{Base case $R=\emp$.}
  Holds because $\CD{\alpha}{\emp}=\emptyset$ and $\mem\models\top$ for any $\mem$.

  \paragraph{Base case $R=\psi Z$}
  Here $R$ is not nullable.
We use the assumption that $\Sigma$ is a set of minterms
which implies that $\den{\psi} = \den{\bigvee\Gamma}$ for some $\Gamma\subseteq\Sigma$
and $\psi\land\alpha$ is unsatisfiable for all $\alpha\in\Sigma\setminus\Gamma$.
\begin{eqnarray*}
  \LANG{\mem}{\psi Z} &=& \den{\psi}\cdot\LANG{\mem}{Z} \\
  &=& \bigcup_{\alpha\in\Sigma}\den{\alpha}\cdot\BraceIfThenElse{\alpha\land\psi\;\textrm{is satisfiable}}{
    \LANG{\mem}{\spair{\id}{Z}}}{\emptyset}\\
  &=& \bigcup_{\alpha\in\Sigma}\den{\alpha} \cdot \LANG{\mem}{\CD{\alpha}{\psi Z}} \\
  &\stackrel{\CanExit{R}=\bot}{=}&
  \LPAR\bigcup_{\alpha\in\Sigma}\den{\alpha} \cdot \LANG{\mem}{\CD{\alpha}{\psi Z}}\RPAR
  \cup \{\eps\mid\mem\models\CanExit{R}\}\\
\end{eqnarray*}

\paragraph{Induction case $R= XZ$ where $X=S\{\ell,k\}$ is a counting loop}
Let $\mem_1 = \IncrOp[X](\mem)$ and let
  $\mem_0=\ExitOp[X](\mem)$.  Observe that $\mem_1=\bot$ iff
  $\mem(X)=k$ and $\mem_0=\bot$ iff $\mem(X)<\ell$.  Note also that if
  $\mem_1=\bot$ then $\mem_0\neq\bot$ because $\ell\leq k$.  So
  $\mem_0$ is valid for $Z$ because $\mem$ is valid for $XZ$.  Also,
  since $\mem$ is valid for $R$, if $S$ contains a counter $c$ then
  $c$ is not visible $R$ and thus $\mem(c)=0$.  Thus, $\mem$ is
  also valid for $S$.

  Assume first that $S$ is not nullable.
  It follows that,
  since $\mem(c)=0$ for all $c\in\COUNTERS{S}$ because $\mem$ is valid for $R$
  and, unless $\CanExit{S}=\bot$, there must be at least one
  counter c such that $\MIN{c}>0$ and $\CanExit{S}$ contains the conjunct
  $\CanExit{c}$ and so
  \begin{equation}
    \label{theorem:CD:eq1}
  \mem\not\models\CanExit{S}
  \end{equation}
It is also true that
  \begin{equation}
    \begin{array}{rcl}
    \label{theorem:CD:eq2}
    \mem_0\neq\bot \;\textrm{and}\;
    \mem_0\models \CanExit{Z} &\Iff& 
    \mem\models\CanExit{X} \;\textrm{and}\;
    \ExitOp[X](\mem) \models \CanExit{Z}  \\
    &\Iff&
    \mem\models\CanExit{XZ}
   \end{array}
  \end{equation}
  because $\VISIBLE{XZ}=\{X\}\cup(\VISIBLE{Z}\setminus\COUNTERS{S})$, so only the counters in
  $\{X\}\cup\COUNTERS{S}$ could interfere (if they occur in the scope of $Z$) but their value is $0$ in $\mem_0$
  by validity of $\mem$.
  Let $$ E=\{\eps\mid\mem\models\CanExit{XZ}\}.$$
  We get the following (observe that if $\mem_0=\bot$, then $
  \LANG{\mem_0}{Z}=\emptyset$):
\begin{eqnarray*}
  \LANG{\mem}{XZ} &=& \LANG{\mem}{S}\cdot\LANG{\mem_1}{XZ} \cup \LANG{\mem_0}{Z}
  \\
  &\stackrel{(2\times\textrm{IH})}{=}& \LPAR \bigcup_{\alpha}\den{\alpha}\cdot\LANG{\mem}{\CD{\alpha}{S}}
  \cup \{\eps\mid\mem\models\CanExit{S}\}
  \RPAR\cdot \LANG{\mem_1}{XZ}
  \cup \\
  && \bigcup_{\alpha}\den{\alpha}\cdot \LANG{\mem_0}{\CD{\alpha}{Z}}
  \cup \{\eps\mid\mem_0\neq\bot\; \textrm{and}\;\mem_0\models\CanExit{Z}\}\\
  &\stackrel{(\textrm{(\ref{theorem:CD:eq1}),(\ref{theorem:CD:eq2})})}{=}&
  \LPAR \bigcup_{\alpha}\den{\alpha}\cdot\LANG{\mem}{\CD{\alpha}{S}}
  \RPAR\cdot \LANG{\mem_1}{XZ}
  \cup \bigcup_{\alpha}\den{\alpha}\cdot \LANG{\mem_0}{\CD{\alpha}{Z}} \cup E 
  \\
  &=& \bigcup_{\alpha}\den{\alpha}\cdot\LPAR\LANG{\mem}{\CD{\alpha}{S}}\cdot\LANG{\mem_1}{XZ}\cup
  \LANG{\mem_0}{\CD{\alpha}{Z}} \RPAR \cup E \\
  &=& \bigcup_{\alpha}\den{\alpha}\cdot\LPAR
  \underbrace{\LANG{\mem}{\CD{\alpha}{S}}\cdot\LANG{\mem}{\pair{\IncrOp[X]}{XZ}}}_{
      \begin{array}{@{}c@{}}
         \textrm{$(\star)$} =\\
        \LANG{\mem}{\CD{\alpha}{S}\otimes\spair{\IncrOp[X]}{XZ}}
    \end{array}
  }
  \cup \LANG{\mem}{\spair{\ExitOp[X]}{\emp}\otimes\CD{\alpha}{Z}} \RPAR \cup E \\
    &=&
  \bigcup_{\alpha}\den{\alpha}\cdot\LANG{\mem}{\CD{\alpha}{S}\otimes\spair{\IncrOp[X]}{XZ} \cup
    \spair{\ExitOp[X]}{\emp}\otimes\CD{\alpha}{Z}} \cup E \\
  &=& \bigcup_{\alpha}\den{\alpha}\cdot \LANG{\mem}{\CD{\alpha}{XZ}} \cup E
\end{eqnarray*}
We show next that $(\star)$ holds. Let $\pair{f}{W}\in\CD{\alpha}{S}$.
It suffices to show that
\[
\LANG{\mem}{\pair{f}{W}}\cdot\LANG{\mem}{\pair{\IncrOp[X]}{XZ}} = \LANG{\IncrOp[X](f(\mem))}{WXZ}
\]
holds which is the same as:
\begin{equation}
  \label{theorem:CD:eq}
\LANG{\IncrOp[X](f(\mem))}{WXZ} =
\LANG{f(\mem)}{W}\cdot\LANG{\IncrOp[X](\mem)}{XZ}
\end{equation}
Since $\mem$ is valid for $WXZ$, $\IncrOp[X](f(\mem))$ is also valid
for $WXZ$ because the potential updates to $\mem$ only affect visible counters.  It follows that
\begin{eqnarray*}
\LANG{\IncrOp[X](f(\mem))}{WXZ} &\stackrel{\textrm{(Lemma~\ref{lemma:RESET})}}{=}&
\LANG{\IncrOp[X](f(\mem))}{W}\cdot
\LANG{\RESET{\VISIBLE{W}}(\IncrOp[X](f(\mem)))}{XZ} \\
&\stackrel{(X\notin\COUNTERS{W})}{=}&
\LANG{f(\mem)}{W}\cdot
\LANG{\RESET{\VISIBLE{W}}(\IncrOp[X](f(\mem)))}{XZ} 
\end{eqnarray*}
We have that $\VISIBLE{W}\subseteq\COUNTERS{W}\subseteq\COUNTERS{S}$ by construction of
derivatives and $f$ only affects values of counters in $\VISIBLE{W}$. Since
$\mem$ is valid for $R$ and $\COUNTERS{S}\subseteq\HIDDEN{R}$ it
follows that $\mem(c)=0$ for all $c\in \VISIBLE{W}$, and
$X\in\VISIBLE{R}\setminus\COUNTERS{S}$.  Therefore
$\RESET{\VISIBLE{W}}(\IncrOp[X](f(\mem))) =
\IncrOp[X](\mem)$, i.e.,
the reset cancels the effect of $f$, and so
\[
\LANG{\RESET{\VISIBLE{W}}(\IncrOp[X](f(\mem)))}{XZ} 
=
\LANG{\IncrOp[X](\mem)}{XZ}
\]
This completes the proof of (\ref{theorem:CD:eq}) and $(\star)$, and thus
the induction case under the condition that $S$ is not nullable.
Under the condition that $S$ is nullable, it follows that $\ell=0$
because $R$ is normalized.  But we can pretend that $S$ \emph{is not
  nullable} because $\ell=0$ makes $X$ nullable and the proof remains
unchanged.
%(Observe that we also know that $\LANG{\mem_1}{X} \subseteq
%\LANG{\mem}{X}$ when $\ell=0$ by using Lemma~\ref{lemma:COUNTING}
%which is a different way \tv{I see that the inclusion holds, but I don't see how
%this helps here: either make the alternative proof idea
%clearer or remove it.} of showing that nullablity of $S$ does not
%matter in $X$ when $\ell=0$.)

\paragraph{Induction case $R= S{*}Z$}
Observe that $\mem\models\CanExit{R}$ iff $\mem\models\CanExit{Z}$
because $S{*}$ is nullable and $\mem(c)=0$ for all $c\in\COUNTERS{S}$.
Assume without loss of generality that $S$ is not nullable.
In this case $\mem\not\models\CanExit{S}$. Let $E= \{\eps\mid \mem\models\CanExit{R}\}$.
\begin{eqnarray*}
  \LANG{\mem}{S{*}Z} &=& \LANG{\mem}{S}^*\cdot \LANG{\mem}{Z}\\
  &=& \LANG{\mem}{S}\cdot\LANG{\mem}{S{*}Z} \cup \LANG{\mem}{Z}\\
  &\stackrel{\textrm{($2\times$IH)}}{=}&
  \LPAR\bigcup_{\alpha}\den{\alpha}\cdot
  \LANG{\mem}{\CD{\alpha}{S}}\cup\{\eps\mid \mem\models\CanExit{S}\}
  \RPAR \cdot \LANG{\mem}{S{*}Z} \cup\\
  &&
  \bigcup_{\alpha}\den{\alpha}\cdot
  \LANG{\mem}{\CD{\alpha}{Z}}\cup\{\eps\mid \mem\models\CanExit{Z}\}
    \\
  &=&
  \bigcup_{\alpha}\den{\alpha}\cdot
  \LANG{\mem}{\CD{\alpha}{S}}
  \cdot \LANG{\mem}{S{*}Z} \cup
  \bigcup_{\alpha}\den{\alpha}\cdot
  \LANG{\mem}{\CD{\alpha}{Z}}\cup E \\
  &=&
  \bigcup_{\alpha}\den{\alpha}\cdot
  \LPAR
  \LANG{\mem}{\CD{\alpha}{S}}
   \cdot \LANG{\mem}{S{*}Z} \cup
  \LANG{\mem}{\CD{\alpha}{Z}}\RPAR\cup E
  \\
  &=&
    \bigcup_{\alpha}\den{\alpha}\cdot
    \LPAR
    \bigcup_{\pair{f}{W}\in\CD{\alpha}{S}}
  \LANG{f(\mem)}{W}
   \cdot \LANG{\mem}{S{*}Z} \cup
   \LANG{\mem}{\CD{\alpha}{Z}}\RPAR\cup E
     \\
  &\stackrel{(\star\star)}{=}&
    \bigcup_{\alpha}\den{\alpha}\cdot
    \LPAR
    \bigcup_{\pair{f}{W}\in\CD{\alpha}{S}}
  \LANG{f(\mem)}{WS{*}Z} \cup
  \LANG{\mem}{\CD{\alpha}{Z}}\RPAR\cup E \\
  &=&
    \bigcup_{\alpha}\den{\alpha}\cdot
    \LPAR
    \LANG{\mem}{\CD{\alpha}{S}\otimes\spair{\id}{S{*}Z}}\cup
    \LANG{\mem}{\CD{\alpha}{Z}}\RPAR\cup E \\
  &=&
    \bigcup_{\alpha}\den{\alpha}\cdot
    \LANG{\mem}{\CD{\alpha}{S}\otimes\spair{\id}{S{*}Z}\cup\CD{\alpha}{Z}}\cup E \\
  &=&
    \bigcup_{\alpha}\den{\alpha}\cdot
    \LANG{\mem}{\CD{\alpha}{S{*}Z}}\cup E \\
\end{eqnarray*}
Equality $(\star\star)$ holds by using Lemma~\ref{lemma:RESET}
because $\RESET{\VISIBLE{W}}(f(\mem))=\mem$
since $\mem(c)=0$ for $c\in\COUNTERS{S}$ and
$\VISIBLE{W}\subseteq \COUNTERS{W}\subseteq\COUNTERS{S}$ by definition of conditional
derivatives.

\paragraph{Induction case $R= (Y_1|Y_2)Z$}
In this case $\mem\models\CanExit{Y_iZ}$ iff
$Y_i$ is nullable and $\mem\models\CanExit{Z}$ because
$\mem(c)=0$ for $c\in\COUNTERS{Y_i}$.
\begin{eqnarray*}
  \LANG{\mem}{(Y_1|Y_2)Z} &=& \LANG{\mem}{Y_1Z}\cup \LANG{\mem}{Y_2Z}\\
  &\stackrel{\textrm{($2\times$IH)}}{=}& \bigcup_{i=1,2}\LPAR\bigcup_{\alpha}\den{\alpha}\cdot
  \LANG{\mem}{\CD{\alpha}{Y_iZ}} \cup\{\eps\mid\mem\models\CanExit{Y_iZ}\}\RPAR\\
  &=&
  \bigcup_{\alpha}\den{\alpha}\cdot
  \LANG{\mem}{\CD{\alpha}{Y_1Z}\cup\CD{\alpha}{Y_2Z}} \cup\{\eps\mid\mem\models\CanExit{Y_1Z}\vee\CanExit{Y_2Z}\}\RPAR\\
  &=&
  \bigcup_{\alpha}\den{\alpha}\cdot
  \LANG{\mem}{\CD{\alpha}{(Y_1|Y_2)Z}} \cup\{\eps\mid\mem\models\CanExit{R}\}\RPAR
\end{eqnarray*}
The last equality uses that
$\mem\models\CanExit{Y_1Z}\vee\CanExit{Y_2Z}$ iff
$\mem\models\CanExit{(Y_1|Y_2)Z}$.
The statement follows by the induction principle.
\end{proof}
 
%==============================================================================
\subsection{Proof of Theorem~\ref{thm:deriv}\label{appThmDeriv}}
%==============================================================================

In the proof of Theorem~\ref{thm:deriv} we refer to the following characterization of the
language of a~state~$q$: 
%
% The following fundamental equation holds by definition and is very useful
% in inductive proofs. It is used below. For all $q\in Q$:
%
\begin{equation}
\label{eq:Lq}
\Lq[\aut]{q} = \LPAR\bigcup_{({q},{\alpha},{p})\in\Delta_\aut}\den{\alpha}\cdot\Lq[\aut]{p}\RPAR \cup \{\eps\mid q\in F\}
% we can get rid of one pair of parenthesis:
% \Lq[\aut]{q} = \{\eps\mid q\in F\} \cup \bigcup_{({q},{\alpha},{p})\in\Delta_\aut}\den{\alpha}\cdot\Lq[\aut]{p}
\end{equation}
%\lh{do we need to number this equation?}
%
(i.e., $\eps \in \Lq[\aut]{q}$ iff $q \in F$).
We write $\Lq{q}$ for $\Lq[\aut]{q}$ when
$\aut$ is clear from the context.

\begin{proof}[Proof of Theorem~\ref{thm:deriv}]
  Let $R$ be fixed.
%  For any state $p \in Q_A$ let
%  $\mem_p$ and $S_p$ denote its components, i.e.,
%  $p=\pair{\mem_p}{S_p}$.
  We prove the following statement by induction over the length of $w$:
  \[
  \forall \pair{\mem}{S}\in Q_A: w\in \Lq[A]{\pair{\mem}{S}} \Iff w \in \LANG{\mem}{S}
  \]

  \paragraph{Base case $w=\eps$}
  Fix $q=\pair{\mem}{S}\in Q_A$. Then $\eps\in\Lq[A]{q}$ iff (by~(\ref{eq:Lq}))
  $q\in F_A$ iff (by definition of $F_{\CA(R)}$)
  $\mem\models \CanExit{S}$ iff (by Theorem~\ref{theorem:CD}) $\eps\in \LANG{\mem}{S}$.
  
  \paragraph{Induction case $w=av$}
  Fix $\pair{\mem}{S}\in Q_A$. Choose the unique $\alpha\in\Sigma$ such that $a\in\den{\alpha}$.
  \begin{eqnarray*}
    av \in  \LANG{\mem}{S} &\stackrel{\textrm{(Theorem~\ref{theorem:CD})}}{\Iff}&
    av \in \den{\alpha}\cdot\LANG{\mem}{\CD{\alpha}{S}} \\
    &\Iff& \exists \pair{f}{T}\in\CD{\alpha}{S}: v\in \LANG{f(\mem)}{T} \\
    &\stackrel{\textrm{(IH)}}{\Iff}& \exists \pair{f}{T}\in\CD{\alpha}{S}: v\in \Lq[A]{\pair{f(\mem)}{T}} \\
    &\Iff& \exists \move{\pair{\mem}{S}}{\alpha}{\pair{f(\mem)}{T}}\in\Delta_A:  v\in \Lq[A]{\pair{f(\mem)}{T}}\\
    &\stackrel{\textrm{(\ref{eq:Lq})}}{\Iff}& av \in\Lq[A]{\pair{\mem}{S}} 
  \end{eqnarray*}

  \iffalse

    av\in \Lq[A]{q} &\stackrel{\textrm{(\ref{eq:Lq})}}{\Iff}&
    v\in \bigcup_{\move{q}{\alpha}{p}}\Lq[A]{p} \\
    &\stackrel{\textrm{(IH)}}{\Iff}&
    v \in \bigcup_{\move{q}{\alpha}{p}}\LANG{\mem_p}{S_p} \\
    &\Iff&
    v \in \bigcup_{}\LANG{\mem_p}{S_p} \\

  \fi
The statement follows by the induction principle.
\end{proof}


%==============================================================================
\subsection{Proof of Theorem~\ref{theorem:SIZE}\label{app:theorem:SIZE}}
%==============================================================================

Let $\CDp{R}$ denote the set of all regexes arising through partial
  derivatives applied recursively starting from a normalized regex $R$.
  Formally, let
  \[
  \CD{}{R}\eqdef \{W\mid \exists\alpha,f:\alpha\in\Minterms{R},\pair{f}{W}\in\CD{\alpha}{R}\},
  \]
  then
  $\CDp{R}$ is the least fixed point of the following equations,
  where $L$ is a set of normalized regexes,
  \[
\CDp{R} = \CD{}{R} \cup \CDp{\CD{}{R}}, \quad \CDp{L} = \bigcup_{R\in L}\CDp{R}.
\]
We first prove the following lemma where given a set of normalized regexes $L$ and
a normalized regex $Z$ we let $LZ$ denote the set $\{WZ\mid W\in L\}$ of normalized regexes.
Observe that $\CDp{R}$ is the set of regexes reached after one or more derivations, 
which may but need not
include $R$ itself, e.g., $\CDp{\texttt{b(ab)\{9\}}} = \{\texttt{(ab)\{9\}}, \texttt{b(ab)\{9\}}\}$ includes the start regex while
$\CDp{\texttt{ab}} = \{\texttt{b},\emp\}$ does not.
We write $S{\diamond}$ for a counting loop $S\{\ell,k\}$ or loop $S{*}$.
\begin{lemma}
  \label{lemma:CDp}
  Let $X$ and $Z$ be any normalized regexes.
  Then $\CDp{XZ}=\CDp{X}Z\cup\CDp{Z}$ and if $X$ is a (counting) loop
  $S{\diamond}$ then $\CDp{X}=\CDp{S}X$.
\end{lemma}
\begin{proof}
  \iffalse
  We first prove that if $X=S{\diamond}$ then $\CDp{X}=\CDp{S}X$.
  In order to do so, we let $\CDp[1]{X}\eqdef \CD{}{X}$ and
  $\CDp[n+1]{X} \eqdef \CD{}{X} \cup \CDp[n]{\CD{}{X}}$ where, for a set $L$ of normalized regexes,
  $\CDp[n]{L} \eqdef \bigcup_{X\in L}\CDp[n]{X}$. We prove by induction over $n$ that for all $S$
  \[
  \CDp[n]{S{\diamond}} = \CDp[n]{S}S{\diamond}
  \]
  The base case $n=1$ follows directly from the definition of $\CD{\alpha}{S{\diamond}}$ (for any $\alpha$).
  For the induction case, $n+1$, we get, by using the definition of $\CD{\alpha}{S{\diamond}}$ (for any $\alpha$),
  \[
  \CDp[n+1]{S{\diamond}} = \CD{}{S{\diamond}} \cup \CDp[n]{\CD{}{S{\diamond}}} =
  \CD{}{S}S{\diamond} \cup \CDp[n]{\CD{}{S}S{\diamond}} \stackrel{\textrm{IH}}{=}
   \CD{}{S}S{\diamond} \cup \CDp[n]{S}{\CD{}{S{\diamond}}}
  \]
  It follows that
  \[
  \CDp{X} = \bigcup_{n\geq 1}\CDp[n]{X} = \bigcup_{n\geq 1}\CDp[n]{S}X = (\bigcup_{n\geq 1}\CDp[n]{S})X = \CDp{S}X
  \]
  \fi
  We prove 
  by induction over $\SIZEOF{X}$ that $\CDp{XZ}=\CDp{X}Z\cup\CDp{Z}$.
  The base case $X=\emp$ follows immediately because
  $\CD{}{\emp}=\emptyset$.
  
  Induction case $X=\psi Y$:
  \begin{eqnarray*}
    \CDp{XZ} &=& \{Y Z\} \cup \CDp{YZ} \\
    &\stackrel{IH}{=}&
    \{Y Z\} \cup \CDp{Y}Z \cup\CDp{Z} \\
    &=& (\{Y\} \cup \CDp{Y})Z \cup \CDp{Z} \\
    &=& \CDp{X}Z\cup\CDp{Z}
  \end{eqnarray*}

  Induction case $X=(X_1|X_2)Y$:
  \begin{eqnarray*}
    \CDp{XZ} &=& \CDp{X_1YZ}\cup\CDp{X_2YZ}\\
    &\stackrel{2\times IH}{=}&
    \CDp{X_1Y}Z \cup \CDp{X_2Y}Z \cup\CDp{Z} \\
    &\stackrel{2\times IH}{=}&
    (\CDp{X_1}Y\cup \CDp{Y})Z \cup (\CDp{X_2}Y\cup \CDp{Y})Z \cup\CDp{Z} \\
    &=&
    (\CDp{X_1}Y \cup \CDp{X_2}Y \cup \CDp{Y})Z \cup\CDp{Z} \\
    &=&
    (\CDp{X_1|X_2}Y \cup \CDp{Y})Z \cup\CDp{Z} \\
     &\stackrel{(\star)}{=}&
    \CDp{X}Z \cup\CDp{Z} 
  \end{eqnarray*}
  In $(\star)$, if $Y=\emp$, the equality holds by definition of
  derivatives of a choice node.  
  If $Y\neq \emp$, then $X_1|X_2$ is
  smaller than $X$, and one can apply the IH on $(X_1|X_2)Y$ with $X_1|X_2$ as $X$ and 
  $Y$ as an instance of the universal variable $Z$ in the lemma.

  Induction case of $X=S{\diamond}Y$
 where $S{\diamond}$ is either a counting loop or a ${*}$-loop.
    The proof step uses the property that, for any normalized $W$,
    \begin{equation}
    \label{lemma:CDp:eq}
    \CDp{S{\diamond}W} = \CDp{S}S{\diamond}W \cup \CDp{W}
    \end{equation}
    holds.
    %Observe that the special case of $W=\emp$ implies that $\CDp{S{\diamond}} = \CDp{S}S{\diamond}$
    % because $ \CDp{\emp}=\emptyset$.
        Equation (\ref{lemma:CDp:eq}) is proved as follows:
    \begin{eqnarray*}
      \CDp{S{\diamond}W} &=& \CDp{SS{\diamond}W} \cup \CDp{W}\\
      &\stackrel{\textrm{(IH)}}{=}& \CDp{S}S{\diamond}W\cup\CDp{S{\diamond}W} \cup \CDp{W}\\
      &\stackrel{\textrm{(lfp)}}{=}& \CDp{S}S{\diamond}W \cup \CDp{W}
    \end{eqnarray*}
    where (lfp) holds because
    $\CDp{S{\diamond}W} \subseteq \CDp{S}S{\diamond}W \cup \CDp{W}$ that can be shown separately.
    It follows that
    \begin{eqnarray*}
      \CDp{XZ} &=& \CDp{S{\diamond}(YZ)}\\
      &\stackrel{(\ref{lemma:CDp:eq})}{=}&
      \CDp{S}S{\diamond}YZ \cup \CDp{YZ} \\
       &\stackrel{IH}{=}&
      \CDp{S}S{\diamond}YZ \cup \CDp{Y}Z \cup \CDp{Z} \\
       &=&
      (\CDp{S}S{\diamond}Y \cup \CDp{Y})Z \cup \CDp{Z} \\
      &\stackrel{(\ref{lemma:CDp:eq})}{=}&
      \CDp{S{\diamond}Y}Z \cup \CDp{Z} \\
      &=&
      \CDp{X}Z\cup\CDp{Z}
    \end{eqnarray*}
    The statement follows by the induction principle.
    Observe that (\ref{lemma:CDp:eq}) implies the second part of the lemma by letting $W=\emp$.
\end{proof}

We now present the proof of the theorem. Recall that $|S|$ denotes the
cardinality of a set $S$, and $\SIZEOF{R}$ is the size of a regex $R$ that
denotes the number of abstract syntax tree nodes of $R$ (up to the case of $R =
\eps$ where the size is zero), and $\LEAFCNT{R}$ denotes the number of
predicates nodes in $R$.

\begin{proof}[Proof of Theorem~\ref{theorem:SIZE}]  
  We first prove, by induction over $\SIZEOF{R}$, that (\ref{eq:theorem:SIZE}) holds.
  \begin {equation}
    \label{eq:theorem:SIZE}
  |\CDp{R}|\leq \LEAFCNT{R}
  \end{equation}
  
 \paragraph{Base case $R=\emp$} Then $\CDp{R} = \emptyset$ and
  $\LEAFCNT{R}=0$, and so (\ref{eq:theorem:SIZE}) holds trivially.

 \paragraph{Induction case $R=\psi Z$} This gives us $\CDp{R}=\{Z\}\cup\CDp{Z}$.
  Thus
%\tv{Not sure how the below was obtained. In particular:\\ $\LEAFCNT{\CDp{R}} =
%\LEAFCNT{\{Z\}\cup\CDp{Z}} \leq \LEAFCNT{Z} + \LEAFCNT{\CDp{Z}}
%\stackrel{IH}{\leq} \LEAFCNT{Z} + \LEAFCNT{Z}$. I do not see how to relate
%this to $\LEAFCNT{R} = \LEAFCNT{\psi Z} = 1 + \LEAFCNT{Z}$. The same seems to
%be the case for $\SIZEOF{.}$.}
  \[
  |\CDp{R}| \leq |\CDp{Z}| + 1 \stackrel{IH}{\leq} \LEAFCNT{Z} + 1 =
  \LEAFCNT{R}.
  \]

%\tv{In the rest, change the appropriate occurrences of $\SIZEOF{.}$ to
%$\LEAFCNT{.}$.}

  \paragraph{Induction case $R=(X|Y)Z$} This gives us $\CDp{R}=\CDp{XZ}\cup\CDp{YZ}$.
  Then, by Lemma~\ref{lemma:CDp},
  \[
  \CDp{XZ}\cup\CDp{YZ} = \CDp{X}Z\cup\CDp{Y}Z\cup\CDp{Z}
  \]
  which implies that (observe that, trivially, $|LZ| = |L|$ for any set $L$ and regex $Z$)
  \[
  |\CDp{R}| \leq |\CDp{X}|+|\CDp{Y}|+|\CDp{Z}|
  \stackrel{\textrm{IH}}{\leq}
  \LEAFCNT{X}+\LEAFCNT{Y}+\LEAFCNT{Z} = \LEAFCNT{R}.
  \]
  \paragraph{Induction case $R=S{\diamond}Z$} Here $\CDp{R} = \CDp{S}S{\diamond}Z \cup\CDp{Z}$ by
  using (\ref{lemma:CDp:eq}) in Lemma~\ref{lemma:CDp}. Thus,
  \[
  |\CDp{R}| \leq |\CDp{S}| + |\CDp{Z}| \stackrel{\textrm{IH}}{\leq}
  \LEAFCNT{S}+\LEAFCNT{Z} =  \LEAFCNT{R}.
  \]
  Equation (\ref{eq:theorem:SIZE}) follows by the induction principle.
  Theorem~\ref{theorem:SIZE} follows from (\ref{eq:theorem:SIZE}) because $Q_{\CA(R)} = \CDp{R} \cup\{R\}$ and thus
  $|Q_{\CA(R)}| \leq |\CDp{R}| + 1 \leq \LEAFCNT{R}+1$.
\end{proof}

%%%%%%%%%%%%%%%%%%%%%%%%%%%%%%%%%%%%%%%%%%%%%%%%%%%%%%%%%%%%%%%%%%%%%%%%%%%%%%%
